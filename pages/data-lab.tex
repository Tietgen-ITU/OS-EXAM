\section{Data lab}

\subsection{Describe your implementation of \code{howManyBits(x)}}
The 'how many bits' task is about counting the minimum amount of bits required to represent a specific integer.
Sadly I did not finish this in time for the hand-in deadline. However, I have figured a way to 
solve the problem. The function performs a binary search for the most significant active bit.
The issue is that we also need to do this for negative numbers where the sign bit is active. In order to 
solve this we then take the bitwise negation of x. Doing so leaves us with a number one short of being the
actual absolute value. We do not want the actual absolute value since there is a risk of overflowing back to a negtive number. 
In order to know how many bits that is needed to represent the integer, we count it at each step.
The binary search algorithm starts in the middle of the bit representation, as the line indicates in figure \ref{fig:hmb-step-1}. If there is an active bit on the left side,
we can skip what is on the right side. That means the amount of bits that we do not need to look at on the 
rigth side will be added to the result as the minimum amount of bits needed to represent the integer. 
When the binary search is finished, we add 1 to the result so that we remember the sign bit.

As an example we have \code{x = 0xA0FFF0F0} which is a negative number.
We find out that it is a negative number by looking at the signed bit. We then take the
logical negation of the number and assign the value to the \code{temp}, so that \code{temp = 0x5f000f0f}. 
Now what we want to do is to figure out where the most significant active bit is. As mentioned,
it can be found by performing binary search.

As the line indicates in figure \ref{fig:hmb-step-1}, we start from the middle:

\centeredpic{howmanybits-01.png}{Bit representation of the negated integer of \code{x}}{hmb-step-1}

We look to the left of the line and check if there is at least 1 bit that is active(i.e, a bit with the value 1). In figure \ref{fig:hmb-step-1}, there is a bit that is active.
For now, we can with confidence say that we at least need 16 bits in order to represent this number. This is due to the
active bit in the left side, so we do not need to look at the right side. 

The following code provides the same result:
\begin{lstlisting}
    int active_bits_left = !!(temp >> 16);
    min_required_bits = active_bits_left << 4;
    
    /* 
    * Shift the result such that we can continue the search for the 
    * last active bit 
    */
    temp >>= min_required_bits;
    result += min_required_bits; // Store the min required bits
\end{lstlisting}

The \code{!!(temp >> 16)} looks at the left half of the bit representation and returns 1 if there is an active bit and 0 if not. We then bit shift it by 4 in order to get the amount of bits 
that we can guarantee that is required to represent the integer. 

At line 8 we bit shift the \code{min\_required\_bits}, i.e. 16 in this case, such that we can continue our search. The result looks like the following:
\centeredpic{howmanybits-02.png}{The bit representation of temp after bits shifting it. Grey blocks have been bit shifted out}{hmb-step-2}

As figure \ref{fig:hmb-step-2} shows, we now continue the process of binary search and take the half again. Again we can see that there is
a bit on the left side of the line. This guarantees that we now at least need 8 bits on top of the previous 16 bits that we have, i.e. 24.

The code to perform this looks nearly the same as in figure \ref{fig:hmb-step-1}:
\begin{lstlisting}
    int active_bits_left !!(temp >> 8);
    min_required_bits =  active_bits_left << 3;

    /* 
    * Shift the result such that we can continue the search for the 
    * last active bit 
    */
    temp >>= min_required_bits;
    result += min_required_bits; // Store the min required bits
\end{lstlisting}

The difference is at line 1 and 2. At line 1 we half the amount of bits that we want to bit-shift with at each step until it reaches zero. And at line 2 we decrement the 
amount of bits we bit shift the result of the \code{active\_bits\_left} by one at each step until it reaches zero. 
Another thing to notice as well is that we continue to add the minimum required bits that we bit-shift out of the \code{temp} at each step.
The following figure is an illustration of continuing the steps described: 
\centeredpic{howmanybits-03.png}{An illustration of the continuation of the same steps. The grey blocks have been bit-shited out}{hmb-step-last}

Before the last step of figure \ref{fig:hmb-step-last}, the minimum required bits is 30. The last step is a bit different. The left side of the line does not have an active bit.
For that reason, the variable \code{minimum\_required\_bits} is 0 and we do not bit shift.
The process of performing binary search is done. However we still need to add the result of \code{temp}. Because if the bit on the right side of the line is active, then that needs to be added.
Due to our binary search and bit-shifting of the binary representation of the integer, then we only add 1 or 0 to our result. 
Before we return the result, we need to add the sign bit. For that reason we add 1 to the result. We return a result of 32 minimum required bits to represent \code{0xA0FFF0F0}.

\subsection{Describe your implementation of \code{tmin(void)}}
The \code{tmin()} task was about creating the minimum number in a two complement bit representation.
Two's complement uses the most significant bit as a sign bit. That is, the bit at the left most position indicates whether 
it is a negative or positive number. If the bit is 1 then the number is a minus. 
The minimum number in a two's complement system is represented by having the most significant bit set to 1 and then the rest of the bits set to 0.
In a 32 bit system it would look like the following:

\begin{figure}[h]
    $1000 \; 0000 \; 0000 \; 0000 \; 0000 \; 0000 \; 0000 \; 0000_2 = -2147483648_{10}$
    \centering
\end{figure}


Following the set of rules for the assignment, it is solved by having a constant which is \code{0x01}
and then bit-shift by 31 positions like the following:
\begin{lstlisting}
    int tmin() {
        return 0x01 << 31;
    }
\end{lstlisting}