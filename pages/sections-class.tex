\section{Topics from the class}

\subsection{A. What is the difference between traps, faults and aborts in the context of interrupts?}
Traps, faults and aborts are different exceptions that happens in an operating system.

A \textit{trap} is an intentional exception. What happens is that a system call triggers the \textit{trap} and passes control 
to an exception handler. The exception handler then handles the trap and returns to the instruction that is after the system call.
\textit{Traps} are typically used for calls that request services from the kernel. This could be reading a file with the system call \code{read}
or creating a new process with \code{fork}.

A \textit{fault} happens when an instruction causes an error. When this occurs, it then passes control to an exception handler for that specific fault.
Here two things can happen. The exception handler can handle the fault and give back control to the application by re-executing the instruction. 
It could also choose to abort instead and terminate the application that caused the fault.

An \textit{abort} is when the function reaches a fatal error. Again, we pass control to an exception handler.
However, the exception handler does not try to repair the situation. Instead it is being sent to the abort routine 
never giving control back to the application.

\subsection{B. What is the difference between an ephemeral and a well-known port?}
The ephemeral ports are something that is being assigned automatically by the kernel of the client. The operating system has a predefined range of ports that it uses to automatically provide a port when a client application wants to communicate to another service. 
On the other hand, well known ports are typically associated with some service and has been agreed upon by the community. Well-known ports are used as a fixed port for particular services. As an example the port 80 is a well known port for the \textit{http} protocol is used to expose a web service. 

\subsection{C. What is a memory leak?}
A memory leak is when something that is allocated is never cleaned up while the application is running. 
It is a lethal thing for an application to occur. This normally happens when we allocate some data in the heap
by using \code{malloc}, from the standard library, and forgets to use the function \code{free} to clean up the memory
at that specific location.

\begin{lstlisting}
    int do_something() {
        int *numbers_p = malloc(sizeof(int)*10);

        // Init numbers
        for(int i = 0; i < 10; i++) {

            numbers_p[i] = i + 1;
        }

        int sum = get_sum(numbers_p, 10);
        
        return sum;
    }

    int get_sum(int *numbers, int count) {
        int sum = 0;

        for(int i = 0; i < count; i++) {

            sum += numbers[i];
        }

        return sum;
    }
\end{lstlisting}
We can see from the above code that we allocated 10 integers in the heap and initialize them 
to $1, 2, 3 ... 10$. Then the \code{get\_sum()} calculates the sum of all the values and returns it.
At the end of function \code{do\_something()} it returns the \code{sum} value. However, the \code{numbers\_p}
still remains in the heap since we have not called the \code{free()} function.

Having a memory leak in your application indicates that it forgets to clean up.
In order to avoid memory leak we need to keep track of what we have allocated and when we would like to clean it up with the function \code{free()}.

\subsection{D. What is a race-condition?}
A race condition can happen when two concurrent processes change the same variable at nearly the same time. More specifically this can happen if two threads, i.e. thread A and B, tries to perform the following at the same time:
\begin{figure}[h]
    \code{a = a + 1}
    \centering
\end{figure}
An example of running the above code with thread A and B can be seen at figure \ref{fig:race}.
Here \code{a} is a shared variable and is initiated to 0. The issue is that thread A only reads \code{a} before a thread context switch to thread B has happened. Now thread B reads \code{a} as 0 and adds 1 to it. However, when the control is passed back to thread A, thread A still has \code{a} as 0. As figure \ref{fig:race} shows, thread A tries to add 1 to \code{a}. This results in writing 1 to \code{a}, even though the expected result should have been 2. 
\centeredpic[0.5]{Race condition.png}{A table showing a race condition between two threads a and b}{race}

This specific scenario can be very hard to debug. But why? Race conditions does not always occur. So if you were to try debug and recreate the example then another example could occur or it might actually work as expected(in that specific run).
We can use the concept of a \textit{mutex} or a \textit{semaphore}. These two provide operations that, if used correctly, can ensure that 
only one thread at a time can access the critical section at a time. 

It is expensive because even though we have secured the critical section, then all of the threads have to wait until it gets access to
the critical section. In other words, they cannot do any work until it is being let through. This can cause some bottle necks in terms of 
gaining performance if something is taking long time to compute in the critical section.
