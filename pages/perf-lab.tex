\section{Perf lab}

\subsection{A. What is the difference between spatial and temporal locality}
\textit{Spatial locality} is when we access something in memory then it is likely that we reference some of the nearby addresses some time after. An example is
when we loop through arrays. Then we take the next element each time.
\begin{lstlisting}
void print_items(int a[N]) {

    for(int i = 0; i < N; i++) {
        printf("%d \n", a[i]);
    }
}
\end{lstlisting}
As can be seen above, the loop goes through the elements of the array, by accessing it one by one. Here we get the elements
by utilizing how the array structure the data. That is, we get the next item in memory which is from a nearby memory address.
\textit{Temporal locality} is when an item that is being referenced is likely to be 
referenced again in the near future. An example of temporal locality is when we reference a specific variable
at each iteration.
\begin{lstlisting}
int pow(int a, int pow) {
    int tmp = a;

    for(int i = 0; i < pow; i++) {
        tmp = tmp * a;
    }

    return tmp;
}
\end{lstlisting}
\textit{Spatial} and \textit{temporal locality} is very important to limit the amount of cache misses. This is very important in terms of performance.
The better the application utilizes the cache, the faster the application performs. When an application needs a value it first checks if the cache has
the value that we are looking for. If not, it goes further down the memory hierarchy, to get the data that the application needs. 
Since the cache is the closest memory to the CPU then it is also the fastest for the application to use. The difference between main memory and the L1 cache is around 100 clock cycles.
So by utilizing our cache and limit the cache misses with spatial and temporal locality, our applications run faster.

\subsection{B. What is SIMD processing}
SIMD, also known as \textit{Single Instruction Multiple Data}, is a way to perform computations on a lot of data with one instruction. The way that it works is by defining a vector with values that we want to perform computations upon. The \code{gcc} can generate SIMD with AVX instructions by adding the flag \code{-mavx2}(\cite{simd-avx}).

My solution does not utilize SIMD. That is due to not using vectors that we can perform arithmetic operations upon. However, the smooth
solution could have benefitted from it in the cases where we need to take the average of $3 * 3$ pixels. Here we could have created an accumulator vector
and a vector for each row and use the \code{+} between the accumulator and the row. Afterwards we can add each value in the accumulator vector and divide it by nine.
This will lead to removing 3 add instructions in the \code{avg\_smooth} function, making it more performant.
