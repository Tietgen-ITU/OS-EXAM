\section*{Appendix}
\appendix

% Ensure that we add chapters and labels for each appendix under here. If '\Chapter' is used above '\appendix' then it
% will be used as a normal chapter

\section{\code{mm\_malloc() Implementation}} \label{appendix:malloc}
\begin{lstlisting}
/* 
 * Allocates a block of memory with the requested size.
 * 
 * Input:
 * size - The size that should be allocated
 * 
 * Returns:
 * The pointer to the allocated block of memory. 
 * If it is not able to allocate more memory then it returns NULL
 */
void *mm_malloc(size_t size)
{
    size_t asize; /* Adjusted block size */
    size_t extendsize; /* Amount to extend heap if no fit */
    char *bp;

    /* Ignore spurious requests */
    if (size == 0)
        return NULL;

    /* Adjust block size to include overhead and alignment reqs. */
    if (size <= DSIZE)
            asize = 2 * DSIZE; // Sets size to 2 * DSIZE (for header and footer)
    else
        asize = ALIGN(size + DSIZE); 

    /* Search the free list for a fit */
    if ((bp = find_fit(asize)) != NULL) {
        place(bp, asize);
        return bp; 
    }

    /* No fit found. Get more memory and place the block */
    extendsize = MAX(asize,CHUNKSIZE);
    if ((bp = extend_heap(extendsize/WSIZE)) == NULL)
        return NULL;

    place(bp, asize);
    return bp;
}
\end{lstlisting}

\section{\code{find\_fit()} implementation} \label{appendix:find_fit}
\begin{lstlisting}
static void *find_fit(size_t size) {

    // Get index for free list based on size
    int index = get_free_list_index(size); 

    for(int i = index; i < num_free_lists; i++) {

        char *current = free_lists[i];

        if(current == NULL)
            continue;

        // Go through the list and take the first that fits
        while(current != NULL) {

            size_t current_size = GET_SIZE(HDRP(current)); 

            /* If the size of the current free block is greater than the requested
            then return it */ 
            if(current_size >= size)
                return current;
                
            current = NEXT_FREE_BLOCK(current);    
        }
    }

    return NULL;
}
\end{lstlisting}

\section{\code{place()} implementation} \label{appendix:place}

\begin{lstlisting}
/*
 * Place the block of memory by setting the header and footer
 * to be allocated with the specified adjusted size.
 * 
 * If the given block pointer has a size difference greater 
 * than or equal to 2 * DSIZE then it is going to split the
 * block and insert the rest of the splitted block into the 
 * free list
 * 
 * Inputs:
 * bp - pointer to the block of memory that should be placed
 * asize - the size that should be placed
*/
static void place(void *bp, size_t asize) {
    size_t size = GET_SIZE(HDRP(bp));
    size_t size_diff = size - asize;
    unsigned int should_split = size_diff >= 2 * DSIZE;
    
    remove_free_block(bp);

    if(!should_split) {

        PUT(HDRP(bp), PACK(size, 1));
        PUT(FTRP(bp), PACK(size, 1)); 
    } else { // Split bp and inserted the rest into the free list
        PUT(HDRP(bp), PACK(asize, 1));
        PUT(FTRP(bp), PACK(asize, 1));

        void *split_p = NEXT_BLKP(bp);
        PUT(HDRP(split_p), PACK(size_diff, 0));
        PUT(FTRP(split_p), PACK(size_diff, 0));
        insert_free_block(split_p);
    }
}

\end{lstlisting}