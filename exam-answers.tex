\documentclass[11pt]{article}
\title{Operating systems and C - Exam Answers}
\author{Andreas Nicolaj Tietgen - ant@itu.dk}

\begin{document}
\maketitle

\section{Data lab}

\subsection{Describe your implementation of \textit{howManyBits(x)}}
// TODO: Describe something here.

\subsection{Describe your implementation of \textit{tmin(void)}}
The \textit{tmin} tasks was about creating the minimum number in a two complement bit representation.
Two's complement uses the most significant bit as a sign bit. That is, the bit at the left most position indicates whether 
it is a negative or positive number. If the bit is 1 then the number is a minus. 

In terms of how the minimum number is represented in a two's complement system then it is by having the first bit set to 1 and then rest of the bits set to 0.
In a 32 bit system then it would look like the following:

$1000 0000 0000 0000 0000 0000 0000 0000 = -2147483648$

Following the set of rules for the specific assignment, then it could be solved by having a constant which is $0x01$
and then bit-shift by 31 positions.   

\section{Perf lab}

\section{Malloc lab}

\section{Topics from the class}


\end{document}