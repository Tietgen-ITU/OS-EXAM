\documentclass[11pt]{article}
\usepackage{xcolor}
\usepackage[T1]{fontenc}
\usepackage{inconsolata}

\newcommand{\code}[1]{{\colorbox{lightgray!20}{\color{orange}\texttt{#1}}}}
\newcommand{\temp}[1]{{\color{red}#1}}

\title{Operating systems and C - Exam Answers}
\author{Andreas Nicolaj Tietgen - ant@itu.dk}

\begin{document}
\maketitle

\section{Data lab}

\subsection{Describe your implementation of \code{howManyBits(x)}}
// TODO: Describe something here.

\subsection{Describe your implementation of \code{tmin(void)}}
The \code{tmin()} tasks was about creating the minimum number in a two complement bit representation.
Two's complement uses the most significant bit as a sign bit. That is, the bit at the left most position indicates whether 
it is a negative or positive number. If the bit is 1 then the number is a minus. 

In terms of how the minimum number is represented in a two's complement system then it is by having the first bit set to 1 and then rest of the bits set to 0.
In a 32 bit system then it would look like the following:

$1000 0000 0000 0000 0000 0000 0000 0000 = -2147483648$

Following the set of rules for the specific assignment, then it could be solved by having a constant which is \code{0x01}
and then bit-shift by 31 positions.   

\section{Perf lab}

\subsection{A. What is the difference between spatial and temporal locality}

// TODO: Describe the difference

// TODO: This is mandatory. Provide an example of situations where each is important and explain how caching plays a role

\subsection{B. What is SIMD processing}

// TODO: Explain what SIMD processing is
// TODO: Explain whether or not my solution benefits from SIMD (and if not, why not?)
\section{Malloc lab}
\subsection{Explain in detail your implementation of the \code{mm\_malloc} function}

My implementation of the \code{mm\_malloc()} function stays the same as it was in the book.\temp{(reference the book)}
However, the implementation of the \code{find\_fit()} and \code{place()} has changed, which the \code{mm\_malloc()} use. So in order to explain \code{mm\_malloc()},
the we also have to explain those two functions as well.
The implementation of the malloc lab is the segregated free list. 
The way it is implemented is by first creating an implementation of the implicit free list, then explicit free list, 
which then resulted in the segregated free list.
This means that the segregated free list is basically an explicit free list for each equivalence class.

When implementing the \code{mm\_malloc()} function, it first performs some checks of the requested \code{size}. The first check is to see if the size is equal to 0. If that is the case then we have to return \code{NULL}.
Afterwards, we have to adjust the variable \code{size} to be aligned to 8. There are two cases:
\begin{itemize}
    \item When the requested size is below or equal the constant \code{DSIZE}: We set the adjusted size, i.e. \code{asize}, to $2 * DSIZE$.
    \item When the requested size is above the constant \code{DSIZE}: We use the \code{ALIGN} macro with $size + DSIZE$ as an argument.
\end{itemize}

We add \code{DSIZE} to the actual size in the last case to ensure that we have space for the header and footer.
Now that we have adjusted the size we then try to find a fit in our free list. 
We try to find a fit by using the \code{find\_fit} function.

\subsubsection{find\_fit}
The \code{find\_fit()} utilizes the segregated free list implementation by getting the index to the list that contains the size being requested. 
If the list points to \code{NULL} then we increment the index by 1 and gets the next list. 
If there is not found any fit in any of the lists then \code{find\_fit()} function returns \code{NULL}.
If the list has an entry then due to the first fit implementation, then it will take the first that can hold the requested size.

\subsubsection{place}
\temp{(Add code with line numbers)}

\code{place()} starts by removing the free block, that is going to be allocated, from the free list by using the \code{remove\_from\_free\_list()} function. 
Afterwards it decides whether to split or not. \code{place()} is going to split if the size difference between the free block and the requested size is greater than $2 * DSIZE$.
\temp{Why is it that it needs to be greater than $2 * DSIZE$?}
\code{place()} splits by setting the allocated block to the size of \code{asize}. Afterwards it jumps to the next block with \code{NEXT\_BLKP}.
We then set the header and the footer to be the difference, i.e. the variable \code{size\_diff} and use the \code{insert\_free\_block} to insert the splitted block to the segregated free list.

\subsection{What is pointer arithmetic?}
Pointer arithmetics are a way to move the pointer to another virtual memory address. The special thing to remember is that the amount of bytes 
that the address moves is depending on which type of pointer it is e.g. \code{char}, \code{integer}, \code{long}, and etc.
So as an example in a 64-bit word system, an int32 pointer called \code{int\_p} and a long pointer called \code{long\_p} would move at their own respective data-size.
That means that \code{int\_p+1} would move 4 bytes, whereas the \code{long\_p+1} would move 8 bytes.

That is also why that in malloc lab, in order to move a pointer only 1 byte we then cast the pointer as a char and add one just like: 

\code{HDRP(bp) ((char *)bp - WSIZE)}

\section{Topics from the class}

\subsection{A. What is the difference between traps, faults and aborts in the context of interrupts?}

// TODO: Explain the difference between traps, faults and aborts

\subsection{B. What is the difference between an ephemeral and a well-known port?}

// TODO: Explain the difference and give examples

\subsection{C. What is a memory leak?}
// TODO: Describe what a memory leak indicates

// TODO: Describe when it does occur

// TODO: Describe how to avoid it

\subsection{D. What is a race-condition?}

// TODO: Describe what it is

// TODO: Why is it hard to debug?

// TODO: Which instructions can you use to avoid a race-condition

// TODO Why are these instructions expensive?
\end{document}